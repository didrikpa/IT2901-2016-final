\chapter{Introduction}
This is the project report for the course Informatics Project II - IT2901 written the spring semester, 2016 at the Norwegian University of Science and Technology. This report will give the reader an overview of the project. The assignment was to create a mobile application for Vitensenteret.

\section{The Course}
Informatics Project II - IT2901 \cite{course} is a mandatory course in NTNU's informatics bachelor program. The course is the bachelor's thesis for the students attending the informatics program and is usually taken during the sixth and last semester. The purpose behind this course is to give the students practical experience with a real software development  process for a real customer. This includes obtaining experience on the complete software development process. The students are supposed learn techniques within software engineering,including different programming languages, development tools and project management tools.  

\section{The Group}
The group consisted of six students studying informatics at NTNU. All the members had different background and possessing different set of skills.

\begin{itemize}
\item \textbf{Bjor Løyning Byrkjedal:}\\
Has experience with 2D game-development and architectural design in java and c\# using various libraries and frameworks. Has most experience with object-oriented languages.

\item \textbf{David André Årthun Bakke:}\\
Has some experience with web-developing, designing and development. Most comfortable writing Java, back- and frontend. Has also experience with database management systems as well. 

\item \textbf{Didrik Pemmer Aalen:}\\
He is experience with webtechnologies, but primary strengths are backend programming mainly with Java. Is also good at organizing.

\item \textbf{Kristian Svoren:}\\
He has experience with webtechnologies, java, and low level programming. Most comfortable in front-end.

\item \textbf{Nicolas Almagro Tonne:}\\
He is experienced with web technologies like AngularJS, Django, PHP, as well as designing with CSS. Has experience in backend modelling and development as well, but stronger in front-end development. 

\item \textbf{Oscar Thån Conrad:}\\
He has Experience with front-end development and webtechnolgies. Also has experience with group leadership and group management 
\end{itemize}

\section{The Customer}
Vitensenteret\cite{customer} is a popular scientific visitor center in Trondheim. The museum is located in Trondheim city centre and offers many different activities including the main exhibition. Some of them are Robot Lab, planetarium, invention workshops and more. Vitensenterets motto is ``technology and science for everyone!"

\section{Project description}
The initial assignment was describes as follows:\\
\\
``Vitensenteret has made several pathways through the center where you can experience the exhibitions based on themes and interests. We would like to build these pathways into apps for smartphones and combine them with mini-games/ puzzles/ in-depth knowledge etc, that allows visitors to have new experiences in the science center and enrich the experience of the models om display. The pathways will be developed together with the student group."\\
\\
On the first meeting, the customer specified that the main audience for the application should be families with children aged between six and ten. 

\section{The goal}
The goal with this project was to make a fun and interesting application for Vitensenteret. They wanted an application that would help the visitors get through the whole museum in a sensible way. They had seen trends that many visitors did not visit every part of the exhibition. To solve this they wanted an application that made the visitors spread throughout the whole exhibition.

\section{Definitions and abbreviations}
This section contains the definition of all terms, as well as all abbreviations needed to understand this report.

\begin{itemize}
\item \textbf{Vitensenteret}\\
Vitensenteret is the norwegian name for Trondheim Science Museum.
\item \textbf{NTNU}\\
Norges Teknisk Naturvitenskapelige Universitet or Norwegian University of Science and Technology
\item \textbf{JDK}\\
Java Standard Edition development kit
\item \textbf{SDK}\\
Software development kit
\item \textbf{IDE}\\
Integrated development environment
\item \textbf{XAML}\\
Extensible Application Markup Language
\item \textbf{Windows}\\
Microsoft Windows operating system
\item \textbf{Linux}\\
A family of operating systems which are UNIX-like, such as Debian and Ubuntu.
\item \textbf{OS X}\\
Apple's operating system for desktop and laptop computers.
\item \textbf{iOS}\\
Apple's operating system for mobile devices.
\item \textbf{HTML}\\
HyperText Markup Language; a markup language for defining the structure of web pages.
\item \textbf{CSS}\\
Cascading Style Sheets; for styling web pages.
\item \textbf{JS}\\
JavaScript; a high-level, dynamic, untyped, and interpreted programming language for adding dynamic functionality to web pages.
\item \textbf{Java}\\
General-purpose, class-based computer programming language.
\item \textbf{C}\\
Low-level and efficient computer programming language.
\item \textbf{C++}\\
General purpose programming language known for performance with low-level memory manipulation.
\item \textbf{C\#}\\
A multi-paradigm programming language encompassing strong typing, imperative, declarative, functional, generic and object-oriented,  developed by Microsoft.
\item \textbf{Python}\\
A high-level, general-purpose, interpreted, dynamic programming language.
\item \textbf{MVC}\\
Model, View, Controller; a software architectural pattern for implementing user interfaces on computers.
\item \textbf{MVW}\\
Model View Whatever; the same as MVC, but one can use ``Whatever" instead of a controller.
\item \textbf{UI}\\
User interfrace; the space where interactions between humans and machines occur.
\item \textbf{Apache}
The world's most used web server software.
\item \textbf{Angular JS}


\end{itemize}

