\chapter{Evaluation and Conclusion}
This last chapter concludes our work of the final project and gives short summary of how the whole process has gone. In addition to explain what kind of problems the group ran into on their way. Lastly it covers lessons the team has learned and what the learning outcome the course has been. 

\section{The Assignment}
The assignment the group initially given was as follows:\\
\\
"Vitensenteret has made several pathways through the center where you can experience the exhibitions based on themes and interests. We would like to build these pathways into apps for smartphones and combine them with mini-games/ puzzles/ in-depth knowledge etc, that allows visitors to have new experiences in the science center and enrich the experience of the models om display. The pathways will be developed together with the student group."\\
\\
After the first customer meeting the group realized that this was really not what Vitensenteret had in mind. The group was told to create almost whatever that they wanted. As long as they took advantage of new and popular technology. This meant the group could not start developing right away. They first had to define what the assignment was going to be, which led a few unforeseen challenges.\\
\\
With the little experience the group had with defining an assignment it was difficult to decide on an realistic and doable assignment, that both the customer and the group agreed on. This gave the group a two weeks delayed start on starting the planning and development.

\section{Challenges}
During the semester the group encountered many different challenges, other than previously listed. Our biggest problems were illness, injuries, absence and poor communication within the group and with the customer. Some of the challenges below were accounted for in the risk analysis, but not to the correct extent.

\subsection {Illness and injuries}
Unfortunately the group was troubled with a lot of illness throughout the whole working period. Although illness was expected to be a challenge form the beginning, it was a much bigger challenge than what was accounted for in the risk analysis. At some point the group had two members who were sick for almost a week at the same time, which slowed down the work flow considerably. \\
\\
The group also suffered form a few injuries also. This resulted in the fact that some team members who were unable to work for shorter periods of time.

\section{Absence}
Another challenge was absence. Many of the team members had made plans, out of town for the public holidays in May. This made it hard for the team to work together as a group in the last final weeks. 

\subsection{The customer}
The team also had problems communicating with the customer. On several occasions the group requested help and support from the customer, but many misunderstandings and with little follow-up from their side led to the fact that many request were not followed up or done properly. This caused further more delay for the team.

\section{Decisions}
Two major choices were made during the project, which benefited the group greatly. The first decision was to make use of the framework Ionic. At first the team encountered some problems setting it up and understanding how to use it. Looking back it it is clear how it benefited us, and spared us for a lot of work. HVA VAR AVGJØRELSE NUMMER 2 IGJEN?

\section{Customer interaction}
The communication with the customer was done through email. Mainly with Arnfinn, but also with the rest of the people involved with the project. In retrospect email was maybe not the best way of communicating, when looking back on the communication problems the group faced. 

\section{Group interaction}
Most of the group interaction was done through instant text messages on Facebook. Since all the team members actively used Facebook it worked very well. One was able to reach the whole group fast and easy by sending a text message on our group chat. All the group member then instantly got a notification on their cellphone, which made the response time very short.

\TODO{Remove parts of teamwork that are irrelevant and fix "us"}
\subsection{Teamwork}

In the beginning of the project the group decided to have a group meeting every Monday at noon. The group also agreed to meet the customer and the supervisor every other week. Although those meetings were not held in the same weeks. This made it a little bit harder for us to plan our sprints. The group chose to do it this way in order to provide the best service to the customer. Therefore the group planned our sprints based on when our customer meetings were held. 
\\
\\
We have decided that our first and second week will conform to week six and seven in the week numbers of the year. The group had two weeks before week one and two which the group filled with brainstorming for the application. This is because the task the group got from our customer was vaguely defined. The customer had some ideas for the application, but wanted to see what the group could come up with. So the group used two weeks just coming up with ideas, then after two weeks the group had a new customer meeting where our ideas got approved. This opened up for us to start planning the executing of the project. The group used most of week one on planning and finishing the preliminary report.
\\
\\
After only a few weeks the group realized how valuable our group meeting were, and how much work the group were able to get done when the whole group was working together. The group then decided to meet three more times a week just to be able to work together. 
\\
\\
During the third week the group decided to start using Ionic, which is a powerful framework that builds on top of Cordova. This introduced some new knowledge for us to acquire and caused some delay since the group had to get a basic understanding of how it works. This also caused the team to discard a lot of the work the group had already done, because the introduction of Ionic, the work was not compatible with the rest of code.
\\
\\
According to the activity plan that the group made for week 3 and 4, the group were able to finish most of our planned assignments. During week 3, the group had two members that were ill, which is why the group was not able to complete all the planned tasks. 



\section{Lessons learned}
In retrospective one can see that assignment was too comprehensive, which made the workload bigger than expected. This combined with the communication troubles led to the fact that application was not completed 100\%.





From this project we have all 
-bedre kontakt med kunden
-hvordan har vi løst problemer
-invol
- Bestemme seg for kodestopp og holde seg til den
- teste for tidligere versjoner av android og ios

\section{Conclusion}
In conclusion the team is very satisfied with the work the have put down and the application that they have made. The Customer have also given the impression that they are both satisfied and impressed with their new application.\\
\\
In that such a project, the whole team has gained a lot of knowledge in its various areas. This includes everything from development tools, programming languages and how challenging collaboration with a customer may be, several developing theories and methods. The time spent on prestudy and the process around choosing appropriate development method paid of in the end. The team agrees that is has been an interesting and fun project and that the learning outcome has been very rewarding.  

\iffalse må flyttes\fi
When it comes to the group, the atmosphere is very good. The group shares the goal of achieving a good grade. In order to do that the group members have signed a contract which specifies how we are going to work together to reach our goals. 
