\chapter{Further Development}

\section{Eddystone beacons}
In the final application, the group used Google's Eddystone standard for the beacons. It is a standard that works well for us, but it is not implemented to its full potential. Eddystone beacons can provide an URL, from which the user can download content. As long as the phone has latest update of Google Chrome web browser installed, it is possible to get notifications for all Eddystone beacons nearby.
\\\\
Since our application is written using web development languages, it could be ported to a website, and then be viewed on the phone, according to the beacon it connects to. It would then be easier to manage, and update the content of the application, which means that it does not have to be updated in every app-store it is released to.
\\\\
The reason the group did not implement Eddystone from the start, was because the group did not know much about the standard, and Ibeacons was more than enough for our use.


\section{Backend}
The backend with the robot list was not ready for real-world deployment, as it didn't have any moderation of submitted robots, or the necessary functionality for being shown on a large screen on display. It would need automatic scrolling and auto-refresh for showing the newest robots.\\
\\
This was not worked further on as the system would need a form of moderation or filter of the robot names, something Vitensenteret could not provide, and would be too complex to implement rules to filter out inappropriate content in the final days of the project.
\\\\
\section{Map screen}
The map screen was not developed in conjunction with beacon technology. One of the ideas were to implement some sort of highlighting feature to add additional purpose to the map screen. Such a feature could provide an overview over which rooms the user has and hasn't visited yet. This could be done by having a beacon in each room letting the app know if the user has entered, and for how long.

\section{Rewards}
Having a reward at the end of the game which makes the player think "WOW" would be something both the group and Vitensenteret would have wanted, but it was not part of the core functionality the group focused on. When the end of the project neared it was obvious that it could not be done in time, and the idea was scrapped.
\\\\
If there had been more time for development, the group would have connected one of the beacons to a Tesla-coil Vitensenteret said could be used for a ``robotic concert", so that a player who finished the game could activate it. There were also discussions about other forms of rewards, and it was decided that some of the group's members would work further with Vitensenteret to come up with a reward for players who won.

\section{Summer development}
Vitensenteret was very interested in having a completed and published application, and asked the group if they were interested in developing the application further during the summer. Several of the group's members were interested in seeing the project through, and were available during the summer. The persons in question would discuss terms and how it would be conducted further with Vitensenteret after the semester had ended.
\\\\
The main thing which had to be done for a launch of the application was installing all the beacons at the exhibition. The group had received enough beacons to do this from Nordic, so this is something that will be implemented completely during the summer. Other small things needed before the launch was creating a more coherent storyline and create a nicer welcome-picture. These were both things Vitensenteret would work on, and come back to the group with during the summer.