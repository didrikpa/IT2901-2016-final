\section{Functional requirements}


\subsection{Overall functional requirements}
\begin{table}[!h]
\caption{Functional requirement for hints in minigames}
\begin{tabular}{ |p{7cm}|p{7cm}| }
\hline
\textbf {Name of functional requirement} & {Hints in minigames} \\ \hline
\textbf {Priority} & {High} \\ \hline
\textbf {Threat} & { If children get stuck without the possibility to get help or guidance they might give up finishing the game, and perhaps the other games as well.} \\ \hline
\textbf {Requirement} & {The game should provide guidance where necessary. If the user submits an answer that is partially correct, they should get some feedback about this. Or if the user get stuck in a minigame, they should have the possibility to get a hint that can help them finish the minigame.} \\ \hline
\textbf {Action} & {Feedback regarding the submitted answer, or hint-button that give the user hints.} \\ \hline
\textbf {Use case} & {1} \\ \hline
\end{tabular}
\end{table}

\begin{table}[!h]
\caption{Functional requirement for quiz minigame }
\begin{tabular}{ |p{7cm}|p{7cm}| }
\hline
\textbf {Name of functional requirement} & {Quiz} \\ \hline
\textbf {Priority} & {Medium} \\ \hline
\textbf {Threat} & {Visitors might not be motivated to read the questions already posted on the walls of the museum. } \\ \hline
\textbf {Requirement} & {There should be a quiz minigame. The questions in the quiz should be the same questions as the ones posted on the walls in Vitensenteret.} \\ \hline
\textbf {Action} & {Implement the quiz minigame} \\ \hline
\textbf {Use case} & {2} \\ \hline
\end{tabular}
\end{table}

\begin{table}[!h]
\caption{Functional requirement for color lock minigame}
\begin{tabular}{ |p{7cm}|p{7cm}| }
\hline
\textbf {Name of functional requirement} & {Color lock minigame} \\ \hline
\textbf {Priority} & {Medium} \\ \hline
\textbf {Requirement} & {There should be a minigame where the user is provided with a color and three sliders, representing red, green, and blue. When the sliders are moved the color changes accordingly. The user is supposed to replicate 5 color provided by the application.} \\ \hline
\textbf {Action} & {Implement Mimic the color lock minigame} \\ \hline
\textbf {Use case} & {2} \\ \hline
\end{tabular}
\end{table}

\begin{table}[!h]
\caption{Functional requirement for shortest path minigame}
\begin{tabular}{ |p{7cm}|p{7cm}| }
\hline
\textbf {Name of functional requirement} & {Shortest path minigame} \\ \hline
\textbf {Priority} & {Medium} \\ \hline
\textbf {Requirement} & {There should be a shortest path minigame. The game should contain the same cities as the installation in the exhibition. The order should be scrambled, and it is up to the user to drag an drop the cities into the right order} \\ \hline
\textbf {Action} & {Implement the shortest path minigame} \\ \hline
\textbf {Use case} & {2} \\ \hline
\end{tabular}
\end{table}

\begin{table}[!h]
\caption{Functional requirement for elements minigame}
\begin{tabular}{ |p{7cm}|p{7cm}| }
\hline
\textbf {Name of functional requirement} & {Elements minigame} \\ \hline
\textbf {Priority} & {Medium} \\ \hline
\textbf {Requirement} & {There should be a minigame with pictures of things that are mainly made from one element. The user should examine the table of elements in the exhibition to acquire information about what elements the thing on the picture is made of.} \\ \hline
\textbf {Action} & {Implement the elements minigame} \\ \hline
\textbf {Use case} & {2} \\ \hline
\end{tabular}
\end{table}

\begin{table}[!h]
\caption{Functional requirement for pipes minigame}
\begin{tabular}{ |p{7cm}|p{7cm}| }
\hline
\textbf {Name of functional requirement} & {Pipes minigame} \\ \hline
\textbf {Priority} & {Medium} \\ \hline
\textbf {Requirement} & {There should be minigame with a representation of a pipesystem. The user rotates the pipes 90 degrees by tapping them. The game is won by connecting a water source to a turbine.} \\ \hline
\textbf {Action} & {Implement the guide the water through the pipes minigame} \\ \hline
\textbf {Use case} & {2} \\ \hline
\end{tabular}
\end{table}

\begin{table}[!h]
\caption{Functional requirement for melody minigame}
\begin{tabular}{ |p{7cm}|p{7cm}| }
\hline
\textbf {Name of functional requirement} & {Melody minigame} \\ \hline
\textbf {Priority} & {Medium} \\ \hline
\textbf {Requirement} & {There should be a melody minigame. The game should provide the user with a sound sample, which the user is supposed to imitate by placing pipes on an installation in the exhibition in a pattern that recreates the sound.} \\ \hline
\textbf {Action} & {Implement the mimic the melody minigame} \\ \hline
\textbf {Use case} & {2} \\ \hline
\end{tabular}
\end{table}

\begin{table}[!h]
\caption{Functional requirement for Simon says minigame}
\begin{tabular}{ |p{7cm}|p{7cm}| }
\hline
\textbf {Name of functional requirement} & {Simon says minigame} \\ \hline
\textbf {Priority} & {Medium} \\ \hline
\textbf {Requirement} & {There should be a Simon says minigame that requires the user to repeat a series of patterns that are generated randomly.} \\ \hline
\textbf {Action} & {Implement the Simon says minigame} \\ \hline
\textbf {Use case} & {2} \\ \hline
\end{tabular}
\end{table}

\begin{table}[!h]
\caption{Functional requirement for language in the application}
\begin{tabular}{ |p{7cm}|p{7cm}| }
\hline
\textbf {Name of non-functional requirement} & {Language} \\ \hline
\textbf {Priority} & {High} \\ \hline
\textbf {Threat} & {Non-norwegian users could be using the application} \\ \hline
\textbf {Requirement} & {The application should have the option to change language from Norwegian to English and vice versa.} \\ \hline
\textbf {Action} & {Implement a dictionary that contains both Norwegian and English texts.} \\ \hline
\end{tabular}
\end{table}


\begin{table}[!h]\label{map_req}
\caption{Functional requirement for the map}
\begin{tabular}{ |p{7cm}|p{7cm}| }
\hline
\textbf {Name of functional requirement} & {Map} \\ \hline
\textbf {Priority} & {medium} \\ \hline
\textbf {Requirement} & {Because the minigames are somewhat connected to the exhibition there should be a map that tells the user where in the exhibition the different minigames should be completed} \\ \hline
\textbf {Action} & {Implement an easily available map with locations.} \\ \hline
\textbf {Use case} & {3} \\ \hline
\end{tabular}
\end{table}

\begin{table}[!h]
\caption{Functional requirement for collecting parts}
\begin{tabular}{ |p{7cm}|p{7cm}| }
\hline
\textbf {Name of functional requirement} & {Collect robot-parts} \\ \hline
\textbf {Priority} & {High} \\ \hline
\textbf {Requirement} & {For every completed minigame the user should be rewarded with a robot part.} \\ \hline
\textbf {Action} & {Ask Vitensenterets' designer to draw several modular parts that are easily interchangeable. Implement sliders to edit the hue and lightness of the parts.} \\ \hline
\end{tabular}
\end{table}

\begin{table}[!h]
\caption{Functional requirement for editing robot parts}
\begin{tabular}{ |p{7cm}|p{7cm}| }
\hline
\textbf {Name of functional requirement} & {Edit robot parts} \\ \hline
\textbf {Priority} & {Low} \\ \hline
\textbf {Requirement} & {In order for the user to be more invested in their robots, they should be able to edit their appearance by changing the combination of different parts, as well as the colors} \\ \hline
\textbf {Action} & {Implement rewarding “ceremony” after each completed minigame.} \\ \hline
\end{tabular}
\end{table}




\subsection{Detailed functional requirements for melody minigame}
Following are the detailed functional requirements for the melody minigame. Similar requirements were developed for the other minigames, but have not been included in this report.

\begin{table}[!h]
\caption{Functional requirement for number of stages}
\begin{tabular}{ |p{7cm}|p{7cm}| }
\hline
\textbf {Name of functional requirement} & {Melody number of stages} \\ \hline
\textbf {Priority} & {High} \\ \hline
\textbf {Requirement} & {Because the installation in the exhibit has 3 different shapes (triangle, rectangle and pentagon), the game should have the same number of stages} \\ \hline
\end{tabular}
\end{table}

\begin{table}[!h]
\caption{Functional requirement sound}
\begin{tabular}{ |p{7cm}|p{7cm}| }
\hline
\textbf {Name of functional requirement} & {Melody play sound} \\ \hline
\textbf {Priority} & {High} \\ \hline
\textbf {Requirement} & {There should be a button that can be pressed as many times as the user wishes that plays the sound to te corresponding stage. If the button is pressed again before the sound is finished playing, it should restart.} \\ \hline
\end{tabular}
\end{table}

\begin{table}[!h]
\caption{Functional requirement selecting a pipe}
\begin{tabular}{ |p{7cm}|p{7cm}| }
\hline
\textbf {Name of functional requirement} & {Melody select pipe} \\ \hline
\textbf {Priority} & {High} \\ \hline
\textbf {Requirement} & {The game should have correctly aligned number pickers. When the user selects one, they should be asked to select one of the pipes, represented by numbers.} \\ \hline
\end{tabular}
\end{table}

\begin{table}[!h]
\caption{Functional requirement submitting wrong answer}
\begin{tabular}{ |p{7cm}|p{7cm}| }
\hline
\textbf {Name of functional requirement} & {Melody submit wrong} \\ \hline
\textbf {Priority} & {Medium} \\ \hline
\textbf {Requirement} & {When submitting an answer the user should be prompted information regarding its correctness. If none of the pipes are placed correctly this should be communicated to the user. If some of the pipes are placed correctly, the user should be given information about how many and which.} \\ \hline
\end{tabular}
\end{table}

\begin{table}[!h]
\caption{Functional requirement submitting correct answer}
\begin{tabular}{ |p{7cm}|p{7cm}| }
\hline
\textbf {Name of functional requirement} & {Melody submit correct} \\ \hline
\textbf {Priority} & {Medium} \\ \hline
\textbf {Requirement} & {When submitting a correct answer the user should get feedback, and be able to move on to the next stage. If the game is completed they should be awarded their robot-part.} \\ \hline
\end{tabular}
\end{table}